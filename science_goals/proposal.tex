\documentclass[letterpaper,11pt,preprint]{hack_aastex}

\input{dfm_stylez}
\pagestyle{myheadings}
\markright{\textsf{\footnotesize %
                   LSST / %
		   Ruth Angus }}

% Single-spacing.
\def\baselinestretch{1.0}

\begin{document}

\title{Maximizing Science in the LSST era: stellar rotation and
gyrochronology}

\section*{Stellar ages}
Ages are notoriously difficult to infer because stars vary little in brightness
or temperature during their hydrogen-burning lifetimes and fitting stellar
evolution models to these two observables usually produces age estimates with
uncertainties in the order of 50-150\%.
However, gyrochronology, the method of inferring an age from mass (or
suitable proxy, e.g. effective temperature) and rotation period has the
potential to provide precise and accurate ages for stars observed by precise
photometric surveys such as LSST.

Cool stars spin down predictably over their main sequence lifetime due
to magnetic braking and their ages depend on their masses and current rotation
periods \citep[e.g.][]{skumanich, kawaler, barnes}, with metallicity also
having a minor influence.
Observations of young cluster stars show a tight relation between rotation
period and age, \citep{meibom} however until recently the behavior of old
stars was unknown due to a dearth of precise ages for field stars.
Recently, the relatively old \Kepler\ asteroseismic targets were precisely
characterized and these stars seem to tell a different story: they
rotate more rapidly than expected given their age and mass \citep{angus}.
This finding provoked the response of \citet{vansaders} who attribute this
behaviour to an evolving magnetic dynamo.
It is clear that our understanding of the physics driving gyrochronology is
incomplete and this is due to the sparcity of the available data: rotation
periods have been measured for only a handful of relatively young clusters
and fewer than 30 suitable asteroseismic stars.
In order to improve the gyrochronology relations we must expand this data set
and, in particular, it is essential that we measure long rotation periods for
old stars.
LSST will provide precision photometry at appropriate cadences for
measuring the rotation periods of intermediate to slowly rotating stars.
Some of these will be in the halo; since the ages of halo stars are known
to be 10-12 Gyr \citep[e.g.][]{jofre}, their rotation periods will enable us
to calibrate gyrochronology at the old end and to quantify the level of
dispersion in the relation between rotation period and age.

LSST will provide rotation periods for thousands of stars.
In order use these targets to improve the gyrochronology relations, effective
temperatures, metallicities and isochronal ages must be inferred for a subset
of this sample.
Once these parameters have be established for such a `training set', we can
develop a model to assign ages to the remaining ensemble, using the
information available from the survey.

% \ssfigure{../../code/gyro_plot.png}{0.4}{%
% The age-rotation (gyrochronology) relation for Solar-mass stars.
% The blue points show the mean rotation periods of clusters Coma Berenices,
% Praesepe, Hyades, NGC6811 and NGC6819 from left to right.
% The black points are local field stars with asteroseismic ages and photometric
% rotation periods, 18 Sco (left) and $\alpha$ Cen A (right) and the pink
% point is the Sun.
% This figure demonstrates how a rotation period
% % and mass, or mass proxy
% can be used to predict stellar age.
% \label{fig:gyro}}

\begin{multicols}{2}
{\centering\bf REFERENCES\par}
\vspace{0.2em}
\begin{thebibliography}{}%
\raggedright\raggedbottom\scriptsize\setlength{\parskip}{-0.5em}%

\bibitem[Angus \etal(2015)]{angus}
Angus, R., Aigrain, S., Foreman-Mackey, D., McQuillan., A., 2015,

% MNRAS, 450, 1787
% \bibitem[Angus, Aigrain \& Foreman-Mackey(in press)]{AngusIAU}
% Angus, R. Aigrain, S. \& Foreman-Mackey, D., Conference proceedings of
% the International Astronomical Union XXIX, 2015
% \bibitem[Angus \etal(ApJ submitted)]{sip}
% Angus, R., Foreman-Mackey, D. \& Johnson, J., Submitted to ApJ
% \bibitem[Angus \& Kipping(ApJ submitted)]{flicker}
% Angus, R. \& Kipping, D., Submitted to ApJ
% \bibitem[\protect\citeauthoryear{Bastien \etal}{2013}]{bastien}
% Bastien, F. A., Stassun, K. G., Basri, G. \& Pepper, J., 2013, Nature, 500, 427

\bibitem[Barnes(2003)]{barnes}
Barnes, S.~A., 1972, ApJ, 586, 464

% \bibitem[Ballard \& Johnson(2014)]{ballard}
% Ballard, S., \& Johnson, J. A. 2014, arXiv:1410.4192
% \bibitem[Borucki \etal(2010)]{kepler}
% Borucki, W.~J., Koch, D., Basri, G. \etal, 2010, Science, 327, 977
% \bibitem[Burke \etal(2015)]{burke}
% Burke, C.~J., Christiansen, J.~L., Mullaly, F. \etal, 2015, ApJ, 809, 8B
% \bibitem[Davies \etal(2014)]{davies}
% Davies, M.~B., Adams, F.~C., Armitage, P., \etal, 2014,
% Protostars and Planets VI, 787
% \bibitem[Dressing \& Charbonneau(2015)]{dressing}
% Dressing, C.~D. \& Charbonneau, D., 2015, \apj, 807, 45
% \bibitem[Epstein \& Pinsonneault(2013)]{epstein}
% Epstein, C.~R., \& Pinsonneault, M.~H., 2013, ApJ, 780, 159
% \bibitem[Fang \& Margot(2013)]{fang}
% Fang, J., \& Margot, J.~L. 2013, ApJ, 767, 115
% \bibitem[Foreman-Mackey \etal(2015)]{foreman-mackey}
% Foreman-Mackey, D., Hogg, D. \& Morton, T., D., 2015, \apj, 975, 64
% \bibitem[Funk \etal(2010)]{funk}
% Funk, B., \etal, 2010, A\&A, 516, A82
% \bibitem[Howard \etal(2012)]{howard}
% Howard, A.~W., Marcy, G.~W., Bryson, S.~T. \etal, 2012, \apj, 201, 15
% \bibitem[Howell \etal(2014)]{k2}
% Howell, S.~B., \etal\ 2014, \pasp, 126, 398
% \bibitem[Johansen \etal(2012)]{johansen}
% Johansen, A., Davies, M.~B., Church, R.~P., \& Holmelin, V. 2012, ApJ, 758, 39

\bibitem[Jofr{\'e} \& Weiss(2011)]{jofre}
Jofr{\'e}, P., \& Weiss, A.\ 2011, \aap, 533, A59
\bibitem[Kawaler(1988)]{kawaler}
Kawaler, S.~D., 1972, ApJ, 333, 236
\bibitem[Meibom et al.(2015)]{meibom}
Meibom, S., Barnes, S.~A., Platais, I., et al.\ 2015, \nat, 517, 589

% \bibitem[Lissauer, \etal(2011)]{lissauer}
% Lissauer, J.~J., Fabrycky, D.~C., Ford, E.~B., \etal, 2011, Nature, 470, 53
% \bibitem[McQuillan \etal(2014)]{mcquillan2014}
% McQuillan, A., Mazeh, T. \& Aigrain, S., ApJ, 211, 24
% \bibitem[Morton \& Winn(2014)]{morton}
% Morton, T.~D. \& Winn, J.~N., 2014, ApJ, 796, 47
% \bibitem[Petigura \etal(2013)]{petigura}
% Petigura, E.~A., Howard, A.~W. \& Marcy, G.~W., 2013, PNAS, 110, 19273
% \bibitem[Petrovich(2014)]{petrovich}
% Petrovich, C., 2014, ApJ, 799, 27
% \bibitem[Ricker \etal(2014)]{tess}
% Ricker, G.~R., \etal, 2014, SPIE JATIS, 3
% \bibitem[Shara \etal(2014)]{shara}
% Shara, M.~M., Hurley, J.~R., Mardling, R.~A., 2014, arXiv:1411.7061v1
% [astro-ph.EP]

\bibitem[Skumanich(1972)]{skumanich}
Skumanich, A., 1972, ApJ, 171, 565
% \bibitem[Smith \& Lissauer(2009)]{smith}
% Smith, A.~W. \& Lissauer, J.~J., 2009, Icarus, 201, 381
\bibitem[van Saders \etal(2015)]{vansaders}
van Saders \etal, 2015, Nature (in press)

% \bibitem[Veras \etal(2015)]{veras}
% Veras, D., Brown, D.~J.~A., Mustill, A.~J. \& Pollacco, D., 2015, MNRAS, 453,
% 67
% \bibitem[Walkowicz \& Basri(2014)]{walk}
% Walkowicz, L.~M. \& Basri, G.~S., 2014, MNRAS, 436, 1883
% \bibitem[Winn \& Fabrycky(2015)]{winn}
% Winn, J.~N. \& Fabrycky, D.~C., 2015 A\&A, 53, 409
% \bibitem[Zhou \etal(2007)]{zhou}
% Zhou, J.~L., Lin, D.~N. C., \& Sun, Y.~S. 2007, ApJ, 666, 423
\end{thebibliography}
\end{multicols}
\end{document}
