\documentclass[letterpaper,11pt,preprint]{hack_aastex}

\input{dfm_stylez}
\pagestyle{myheadings}
\markright{\textsf{\footnotesize %
                   LSST / %
		   Ruth Angus }}
\author{Ruth Angus, University of Oxford, UK}

% Single-spacing.
\def\baselinestretch{1.0}

\begin{document}

\title{Maximizing Science in the Era of LSST: stellar rotation and
gyrochronology}

The ages of main sequence stars are notoriously difficult to infer because
stars vary little in brightness
or temperature during their hydrogen-burning lifetimes and isochronal ages
can therefore be very imprecise.
Gyrochronology, the method of inferring an age from mass (or suitable
proxy, e.g.\ effective temperature) and rotation period has the
potential to provide ages for stars observed by precise photometric surveys
such as LSST, however it still suffers from being poorly calibrated at late
ages.
LSST will provide observations of thousands of old stars, ideal for
calibrating gyrochronology and it is important that these targets are
characterised.
In particular, their effective temperatures and surface gravities must be
constrained.
Ideally, spectra of a suitable subset of these stars should be obtained and
used to train a model that can convert LSST colours to effective temperatures
and surface gravities for a larger ensemble of targets.

\section*{Scientific Background}
Cool stars spin down predictably over their main sequence lifetime due
to magnetic braking and their ages depend on their masses and current rotation
periods \citep[e.g.][]{skumanich, kawaler, barnes}, with metallicity also
having a minor influence.
Observations of young cluster stars show a tight relation between rotation
period and age, \citep{meibom} however until recently the behavior of old
stars was unknown due to a dearth of precise ages.
Recently, the relatively old \Kepler\ asteroseismic targets were precisely
characterized.
These stars tell a different story: they rotate more rapidly than
expected given their age and mass \citep{angus}.
Although \citet{vansaders} are able to explain this behaviour by including an
evolving magnetic dynamo in their models, it is clear that our understanding
of the physics driving gyrochronology is still incomplete.
This is due to the sparcity of the available data: rotation periods have been
measured for only a handful of relatively young clusters and fewer than 30
suitable asteroseismic stars.
In order to improve the gyrochronology relations we must expand this data set,
particularly towards slowly rotating, old stars.
LSST will provide precision photometry at appropriate cadences for
measuring the rotation periods of intermediate to slowly rotating stars.
Some of these will be in the halo.
Since the ages of halo stars are known to be 10-12 Gyr \citep[e.g.][]{jofre},
their rotation periods will be particularly useful for calibrating
gyrochronology at the old end and quantifying the level of dispersion in the
relation between rotation period and age.

In order to perform this calibration, spectroscopic effective temperatures and
surface gravities must be inferred for a subset of this sample.
Once temperature and log $g$ has been measured from the spectra of such a
`training set', we will be able to develop a model that converts LSST
observables to temperatures and log $g$s for the remainder of the survey's
targets stars.
We therefore need to outline the requirements for spectroscopic survey that is
large enough to adequately train a model for the characterisation of the
remaining LSST targets.

\begin{multicols}{2}
{\centering\bf REFERENCES\par}
\vspace{0.2em}
\begin{thebibliography}{}%
\raggedright\raggedbottom\scriptsize\setlength{\parskip}{-0.5em}%

\bibitem[Angus \etal(2015)]{angus}
Angus, R., Aigrain, S., Foreman-Mackey, D., McQuillan., A., 2015,

% MNRAS, 450, 1787
% \bibitem[Angus, Aigrain \& Foreman-Mackey(in press)]{AngusIAU}
% Angus, R. Aigrain, S. \& Foreman-Mackey, D., Conference proceedings of
% the International Astronomical Union XXIX, 2015
% \bibitem[Angus \etal(ApJ submitted)]{sip}
% Angus, R., Foreman-Mackey, D. \& Johnson, J., Submitted to ApJ
% \bibitem[Angus \& Kipping(ApJ submitted)]{flicker}
% Angus, R. \& Kipping, D., Submitted to ApJ
% \bibitem[\protect\citeauthoryear{Bastien \etal}{2013}]{bastien}
% Bastien, F. A., Stassun, K. G., Basri, G. \& Pepper, J., 2013, Nature, 500, 427

\bibitem[Barnes(2003)]{barnes}
Barnes, S.~A., 1972, ApJ, 586, 464

% \bibitem[Ballard \& Johnson(2014)]{ballard}
% Ballard, S., \& Johnson, J. A. 2014, arXiv:1410.4192
% \bibitem[Borucki \etal(2010)]{kepler}
% Borucki, W.~J., Koch, D., Basri, G. \etal, 2010, Science, 327, 977
% \bibitem[Burke \etal(2015)]{burke}
% Burke, C.~J., Christiansen, J.~L., Mullaly, F. \etal, 2015, ApJ, 809, 8B
% \bibitem[Davies \etal(2014)]{davies}
% Davies, M.~B., Adams, F.~C., Armitage, P., \etal, 2014,
% Protostars and Planets VI, 787
% \bibitem[Dressing \& Charbonneau(2015)]{dressing}
% Dressing, C.~D. \& Charbonneau, D., 2015, \apj, 807, 45
% \bibitem[Epstein \& Pinsonneault(2013)]{epstein}
% Epstein, C.~R., \& Pinsonneault, M.~H., 2013, ApJ, 780, 159
% \bibitem[Fang \& Margot(2013)]{fang}
% Fang, J., \& Margot, J.~L. 2013, ApJ, 767, 115
% \bibitem[Foreman-Mackey \etal(2015)]{foreman-mackey}
% Foreman-Mackey, D., Hogg, D. \& Morton, T., D., 2015, \apj, 975, 64
% \bibitem[Funk \etal(2010)]{funk}
% Funk, B., \etal, 2010, A\&A, 516, A82
% \bibitem[Howard \etal(2012)]{howard}
% Howard, A.~W., Marcy, G.~W., Bryson, S.~T. \etal, 2012, \apj, 201, 15
% \bibitem[Howell \etal(2014)]{k2}
% Howell, S.~B., \etal\ 2014, \pasp, 126, 398
% \bibitem[Johansen \etal(2012)]{johansen}
% Johansen, A., Davies, M.~B., Church, R.~P., \& Holmelin, V. 2012, ApJ, 758, 39

\bibitem[Jofr{\'e} \& Weiss(2011)]{jofre}
Jofr{\'e}, P., \& Weiss, A.\ 2011, \aap, 533, A59
\bibitem[Kawaler(1988)]{kawaler}
Kawaler, S.~D., 1972, ApJ, 333, 236
\bibitem[Meibom et al.(2015)]{meibom}
Meibom, S., Barnes, S.~A., Platais, I., et al.\ 2015, \nat, 517, 589

% \bibitem[Lissauer, \etal(2011)]{lissauer}
% Lissauer, J.~J., Fabrycky, D.~C., Ford, E.~B., \etal, 2011, Nature, 470, 53
% \bibitem[McQuillan \etal(2014)]{mcquillan2014}
% McQuillan, A., Mazeh, T. \& Aigrain, S., ApJ, 211, 24
% \bibitem[Morton \& Winn(2014)]{morton}
% Morton, T.~D. \& Winn, J.~N., 2014, ApJ, 796, 47
% \bibitem[Petigura \etal(2013)]{petigura}
% Petigura, E.~A., Howard, A.~W. \& Marcy, G.~W., 2013, PNAS, 110, 19273
% \bibitem[Petrovich(2014)]{petrovich}
% Petrovich, C., 2014, ApJ, 799, 27
% \bibitem[Ricker \etal(2014)]{tess}
% Ricker, G.~R., \etal, 2014, SPIE JATIS, 3
% \bibitem[Shara \etal(2014)]{shara}
% Shara, M.~M., Hurley, J.~R., Mardling, R.~A., 2014, arXiv:1411.7061v1
% [astro-ph.EP]

\bibitem[Skumanich(1972)]{skumanich}
Skumanich, A., 1972, ApJ, 171, 565
% \bibitem[Smith \& Lissauer(2009)]{smith}
% Smith, A.~W. \& Lissauer, J.~J., 2009, Icarus, 201, 381
\bibitem[van Saders \etal(2015)]{vansaders}
van Saders \etal, 2015, Nature (in press)

% \bibitem[Veras \etal(2015)]{veras}
% Veras, D., Brown, D.~J.~A., Mustill, A.~J. \& Pollacco, D., 2015, MNRAS, 453,
% 67
% \bibitem[Walkowicz \& Basri(2014)]{walk}
% Walkowicz, L.~M. \& Basri, G.~S., 2014, MNRAS, 436, 1883
% \bibitem[Winn \& Fabrycky(2015)]{winn}
% Winn, J.~N. \& Fabrycky, D.~C., 2015 A\&A, 53, 409
% \bibitem[Zhou \etal(2007)]{zhou}
% Zhou, J.~L., Lin, D.~N. C., \& Sun, Y.~S. 2007, ApJ, 666, 423
\end{thebibliography}
\end{multicols}
\end{document}
